\documentclass[12pt]{article}
\usepackage[T1]{fontenc}
\usepackage[utf8]{inputenc}
\usepackage[francais]{babel}
\usepackage{listings}

\title{Rapport groupe TM7B}

\author{NAVARRO Robin, BOITELLE Adrien}


\begin{document}
\maketitle

\newpage
\tableofcontents\documentclass[12pt]{article}
\usepackage[T1]{fontenc}
\usepackage[utf8]{inputenc}
\usepackage[francais]{babel}
\usepackage{listings}

\title{Rapport groupe TM7B}

\author{NAVARRO Robin, BOITELLE Adrien}


\begin{document}
\maketitle

\newpage
\tableofcontents
\newpage

\section{Introduction}
Pour réaliser ce projet, nous nous sommes beaucoup appuyé à la fois sur le jeu existant, sur son architecture et sur le code du serveur IRC précédemment réalisé en TP. Dans ce rapport nous allons présenter le travail effectué, les méthodes utilisées, le résultat produit ainsi que les possibles améliorations.

\section{Vue d'ensemble fonctionnelle}

Cette section est destinée aux potentiels utilisateurs afin de présenter le projet, guider l'utilisation et alerter des problèmes connus.

\subsection{Utilisation}

Des scripts sont à disposition pour tester les mécanismes de ce projet:
\begin{lstlisting}[language=bash]
  $ ./run_servers.sh
  $ ./run_players.sh
\end{lstlisting}
Si ceux-ci ne fonctionnent pas, remplacer python3 par python. Sinon, vérifier que pygame est bien installé.

Pour démarrer un serveur manuellement ou en rejoindre un, les commandes de base fonctionnent : 
\begin{lstlisting}[language=bash]
  $ python3 bomber_server.py {port} {map_path}
  $ python3 bomber_client.py {host} {port} {username}
\end{lstlisting}

Attention cependant, pour que le téléporteur fonctionne, il est nécessaire d'utiliser les ports 7777 et 7778.
\subsection{Fonctionnalités}
Une fois connecté, un cadre rouge permet d'identifier le personnage contrôlé. Une immunité de 5 secondes est accordée au nouveaux venus. Chaque explosion retire 10 point de vie et chaque fruit mangé en rend 10. Les points de vie du joueur sont visibles en haut à gauche de l'écran et le pavé directionnel permet de déplacer le personnage dans le jeu. Appuyer sur espace permet de poser une bombe à l'emplacement du joueur (il y a un délai de 2 secondes avant de pouvoir poser une autre bombe).
\newpage
Une case particulière, représentée en violet dans le jeu, permet de se téléporter vers une autre serveur.
Dans le cas où cet autre serveur ne serait pas accessible, la téléportation ne se fait pas.
Des bombes apparaissent aléatoirement sur le terrain, afin d'augmenter la difficulté.

\subsection{Problèmes connus}
Malgré nos efforts, quelque problèmes persistent.\\
Au lancement du jeu, aucune vérification n'est faite afin de vérifier que la grille de jeu corresponde à celle du serveur. Nous avons pensé à effectuer une verification de l'intégrité du fichier du terrain puis dans le cas d'un echec à son remplacement, mais cela semblait un peu en dehors des attentes.
\\

En cas de déconnexion du serveur, les clients sont immédiatement fermés. Il n'y a pas de mécanisme permettant de relancer le jeu automatiquement, ou de reprendre la partie.
Par contre, si les joueurs ont utilisé un téléporteur et se trouvent sur un serveur qui n'a pas fermé, il peuvent continuer à jouer. Le téléporteur ne fonctionnera tout simplement plus tant que l'autre serveur ne sera pas à nouveau disponible.
\\

Au cours de développement, il nous est arrivé de rencontrer quelques fonctionnements anormaux, mais impossible de les reproduire.\\
Par exemple, il est arrivé une fois que les points de vies d'un joueur soient différents de 10 unités entre les clients et le serveur. Nous n'avons pas réussi à reproduire cet effet, nous ne pouvons donc que spéculer sur la nature du dysfonctionnement. \\

L'hypothèse la plus probable est qu'un joueur esquive une bombe in-extremis de son point de vue, mais que son mouvement ne soit pas envoyé au serveur et aux autres clients à temps à cause de la latence.
La solution envisagée serait que les clients ne décident plus de retirer de la vie aux autres joueurs, mais que ce soit le serveur qui leur en donnent l'indication.
\newpage

\section{Architecture et Implémentation}
Nous avons implémenté nos modifications en tentant au maximum de respecter la structure existante. L'architecture est donc restée identique à celle proposée.

\subsection{Étude du modèle existant}
Le jeu tel que nous l'avons récupéré utilise le modèle Modèle Vue Controleur. Le controleur avait été modifié pour réaliser une intéraction entre clients et serveurs, tandis que le modèle est resté identique. La vue n'a également pas été modifiée, mais elle n'est pas invoquée complètement par le serveur.
\\

Pour son interface graphique, ce jeu utilise pygame qui est une librairie libre et open source destinée à créer des jeux en python, reposant sur la librairie SDL.
\\

Pour générer les terrains, un format texte est utilisé. Il est décodé dans la vue pour en faire un objet map. Pour que le jeu soit opérant, il faut que chacun ait les mêmes fichiers. Aucune vérification n'est imposée donc il est tout à fait possible qu'un joueur rejoigne une partie avec un fichier de terrain différent, à ses risques et périls.
\subsection{Protocole réalisé}
Afin que la communication s'effectue correctement, nous avons mis au point un protocole simple.
Ce protocole se situe au niveau des couches 7 et 6 du modèle OSI, c'est à dire la couche application et la couche présentation. Pour le transport, nous utilisons le protocole TCP.
Le protocole UDP ne nous permettais pas d'être sure que les paquets ne se soient pas perdus, se qui aurait impliqué de mettre en place un système pour éviter la désynchronisation des joueurs.
TCP nous épargnes ses problèmes en gérant le renvoie des paquets perdus.
\\

Pour la partie présentation du protocole, il faut respecter ce format:\\
\begin{lstlisting}
BEGIN <size> <data>
\end{lstlisting}
Size étant la taille de message, que nous avons arbitrairement limité à 5 octets, data le message à envoyer. Le tout doit être encodé en utf-8.
Ceci nous permet d'avoir une réception non bloquante des données. En effet, nous attendons de recevoir BEGIN, puis la taille, puis ensuite nous recevons la quantité exacte de donnée.
Avant ce mécanisme, nous avions un système d'acquittement pour éviter les réceptions bloquantes. Le serveur attendait un "OK" du client pour envoyer la suite.
Mais le client ne pouvait être sure que le serveur avait reçu le OK. Le système avec la taille des données semblait alors plus pratique à mettre en oeuvre.
\\

Pour communiquer avec le serveur, le client dispose de différentes commandes à envoyer au serveur.\\
\begin{description}
\item[JOIN <nickname>] Permet de rejoindre le serveur avec un pseudonyme.
\item[JOSP <healt> <kind> <nickname>] Permet de rejoindre un serveur en précisant le niveau de vie, le type de personnage et le pseudonyme. Sert pour la téléportation.
\item[MOVE <up/down/left/right>] Permet de bouger son personnage sur le serveur.
\item[DROP] Permet de lancer une bombe.
\item[QUIT] Permet d'annoncer que l'on quitte le serveur.
\end{description}
Le serveur dispose de ses propres commandes afin de communiquer avec les clients :\\
\begin{description}
\item[WELC <nick> <healt> <kind> <x> <y>] Donne au joueur ses informations tels que son type, sa vie, et (x, y) ses coordonnées de départ dans le jeu.
\item[NEWP <nickname> <healt> <kind> <x> <y>] Indique au client la création d'un joueur avec healt points de vie, de type kind, à la position (x, y)
\item[NEWF <kind> <x> <y>] Indique la création d'un fruit de type kind à la position (x, y)
\item[MOVP <nickname> <up/down/left/right] Indique que le joueur <nickname> à bougé dans une certaine direction
\item[QUIT <nickname>] Indique que le joueur <nickname> a quitté la partie
\item[DROP <nickname>] Indique que le joueur <nickname> a posé une bombe.
\item[SERVDROP <x> <y>] Indique que le serveur à posé une bombe à la position (x, y)
\item[TPSP <host> <port>] Indique au client qu'il doit se connecter au serveur <host> sur le port <port>
\end{description}

\subsection{Implémentation du mode multijoueur}

La majeure partie du projet se déroule dans cette implémentation ; Le client et le serveur disposent chacun d'un controleur différent. Pour dialoguer, ils utilisent le protocole décrit dans la partie précédente. Les commandes reçues et envoyées sont généralement composées d'un mot-clé et de paramètres, ce qui va permettre au receveur de la décomposer puis d'appeller la méthode appropriée.
\\

Lorsqu'un client se connecte à un serveur, il lui envoie uniquement son nom d'utilisateur. Le serveur va générer un personnage, l'ajouter à son modèle, puis répondre par un message de bienvenue qui comprend les caractéristiques du personnage généré. Le serveur va également envoyer une commande à tous les autres clients pour que ceux-ci ajoutent à leur tour le personnage. En cas de nom d'utilisateur déjà utilisé, le client est simplement refusé et affiche le message d'erreur adéquat.
\\

Lorsqu'un utilisateur appuie sur une touche de déplacement, elle est directement impactée sur son jeu pour plus de fluidité. La commande est ensuite envoyée au serveur, qui va l'interprêter puis prévenir à leur tour les autres clients. Cette méthode ne provoque pas de désynchronisation, car les pertes de paquets sont gérées par TCP. Cela marche également comme ça pour toutes les autres intéractions.
\newpage
Le jeu se déroule donc avec chaque partie gérant le modèle de leur côté, suivant les instructions du serveur et leurs propres mouvements. Ce système n'est pas du tout sécurisé car même si le code était compilé, un client peut envoyer toutes les instructions qu'il veut. Il existe notament une faille qui sera expliquée dans la section bonus.

\subsection{Bugs et résolution de bugs}

\subsection{Ajout de bonus}
Pour que le jeu soit plus intéressant / jouable, nous avons rajouté l'affichage de la vie en haut à gauche de l'écran. L'image de coeur est un simple pixel art et le texte est écrit avec un module de pygame. Nous avons également rajouté l'apparition de bombes aléatoirement sur le terrain, pour ajouter un peu de difficulté. 
\\

Nous avons également rajouté un teleporteur comme proposé. Le téléporteur est une nouvelle case représentée par le chiffre 3 dans le fichier de terrain, et par une version violette de la case "départ" disponible dans les fichiers image. Lorsqu'un joueur se déplace dessus, il le serveur lui transmet une nouvelle adresse et un nouveau port, où il va ensuite s'auto transférer. La faille du point de vue gameplay est qu'un client malicieux peut envoyer ces requêtes quand il le veut. Ce qui rend cette faille importante est que le joueur peut décider de ses caractéristiques (apparence et points de vie), mais cela a rendu l'implémentation plus simple.
\\

Dans cette implémentation, on conserve facilement les attributs du joueur et si le code n'est pas modifié, cela marche très bien. Si nous voulons tout de même corriger cela, nous l'aurions fait de manière à ce que ce soit le serveur qui communique avec l'autre serveur à propos des caractéristiques du joueur. Il aurait fallu, cependant, faire en sorte que personne ne se fasse passer pour le serveur et pour cela on peut penser à un système de clés publiques clés privées, puis un dictionnaire de serveurs de confiance ... Mais pour ce projet il ne nous semblait pas utile d'aller aussi loin.
\newpage

\section{Bilan}
Réaliser ce projet fut une expérience enrichissante qui nous a permis de réfléchir à des problématiques à côté desquelles nous passons souvent à côté lorsque nous développons des applications en standalone. Il fut également l'occasion de mettre en pratique par nous même des principes vus lors du semestre de Réseau.
\\

Le résultat que nous avons fourni nous satisfait et nous sentons que si une demande de nouvelle fonctionnalités surgissait, ce ne serait pas à contre coeur que nous nous remettrions à travailler dessus, même si il fallait pour cela revenir sur la majorité du code. En attendant, nous avons eu le plaisir de tester ce jeu auquel nous avons rajouté la dimension multijoueur depuis un environnement local, à distance sur un ordinateur personnel puis sur un vrai serveur, afin de constater qu'il est pleinement opérationnel et jouable.
\end{document}
\newpage

\section{Introduction}
Pour réaliser ce projet, nous nous sommes beaucoup appuyé à la fois sur le jeu existant, sur son architecture et sur le code du serveur IRC précédemment réalisé en TP. Dans ce rapport nous allons présenter le travail effectué, les méthodes utilisées, le résultat produit ainsi que les possibles améliorations.

\section{Vue d'ensemble fonctionnelle}

Cette section est destinée aux potentiels utilisateurs afin de présenter le projet, guider l'utilisation et alerter des problèmes connus.

\subsection{Utilisation}

Des scripts sont à disposition pour tester les mécanismes de ce projet:
\begin{lstlisting}[language=bash]
  $ ./run_servers.sh
  $ ./run_players.sh
\end{lstlisting}
Si ceux-ci ne fonctionnent pas, remplacer python3 par python. Sinon, vérifier que pygame est bien installé.

Pour démarrer un serveur manuellement ou en rejoindre un, les commandes de base fonctionnent : 
\begin{lstlisting}[language=bash]
  $ python3 bomber_server.py {port} {map_path}
  $ python3 bomber_client.py {host} {port} {username}
\end{lstlisting}

Attention cependant, pour que le téléporteur fonctionne, il est nécessaire d'utiliser les ports 7777 et 7778.
\subsection{Fonctionnalités}
Une fois connecté, un cadre rouge permet d'identifier le personnage contrôlé. Une immunité de 5 secondes est accordée au nouveaux venus. Chaque explosion retire 10 point de vie et chaque fruit mangé en rend 10. Les points de vie du joueur sont visibles en haut à gauche de l'écran et le pavé directionnel permet de déplacer le personnage dans le jeu. Appuyer sur espace permet de poser une bombe à l'emplacement du joueur (il y a un délai de 2 secondes avant de pouvoir poser une autre bombe).
\newpage
Une case particulière, représentée en violet dans le jeu, permet de se téléporter vers une autre serveur.
Dans le cas où cet autre serveur ne serait pas accessible, la téléportation ne se fait pas.
Des bombes apparaissent aléatoirement sur le terrain, afin d'augmenter la difficulté.

\subsection{Problèmes connus}
Malgré nos efforts, quelque problèmes persistent.\\
Au lancement du jeu, aucune vérification n'est faite afin de vérifier que la grille de jeu corresponde à celle du serveur. Nous avons pensé à effectuer une verification de l'intégrité du fichier du terrain puis dans le cas d'un echec à son remplacement, mais cela semblait un peu en dehors des attentes.
\\

En cas de déconnexion du serveur, les clients sont immédiatement fermés. Il n'y a pas de mécanisme permettant de relancer le jeu automatiquement, ou de reprendre la partie.
Par contre, si les joueurs ont utilisé un téléporteur et se trouvent sur un serveur qui n'a pas fermé, il peuvent continuer à jouer. Le téléporteur ne fonctionnera tout simplement plus tant que l'autre serveur ne sera pas à nouveau disponible.
\\

Au cours de développement, il nous est arrivé de rencontrer quelques fonctionnements anormaux, mais impossible de les reproduire.\\
Par exemple, il est arrivé une fois que les points de vies d'un joueur soient différents de 10 unités entre les clients et le serveur. Nous n'avons pas réussi à reproduire cet effet, nous ne pouvons donc que spéculer sur la nature du dysfonctionnement. \\

L'hypothèse la plus probable est qu'un joueur esquive une bombe in-extremis de son point de vue, mais que son mouvement ne soit pas envoyé au serveur et aux autres clients à temps à cause de la latence.
La solution envisagée serait que les clients ne décident plus de retirer de la vie aux autres joueurs, mais que ce soit le serveur qui leur en donnent l'indication.
\newpage

\section{Architecture et Implémentation}
Nous avons implémenté nos modifications en tentant au maximum de respecter la structure existante. L'architecture est donc restée identique à celle proposée.

\subsection{Étude du modèle existant}
Le jeu tel que nous l'avons récupéré utilise le modèle Modèle Vue Controleur. Le controleur avait été modifié pour réaliser une intéraction entre clients et serveurs, tandis que le modèle est resté identique. La vue n'a également pas été modifiée, mais elle n'est pas invoquée complètement par le serveur.
\\

Pour son interface graphique, ce jeu utilise pygame qui est une librairie libre et open source destinée à créer des jeux en python, reposant sur la librairie SDL.
\\

Pour générer les terrains, un format texte est utilisé. Il est décodé dans la vue pour en faire un objet map. Pour que le jeu soit opérant, il faut que chacun ait les mêmes fichiers. Aucune vérification n'est imposée donc il est tout à fait possible qu'un joueur rejoigne une partie avec un fichier de terrain différent, à ses risques et périls.
\subsection{Protocole réalisé}
Afin que la communication s'effectue correctement, nous avons mis au point un protocole simple.
Ce protocole se situe au niveau des couches 7 et 6 du modèle OSI, c'est à dire la couche application et la couche présentation. Pour le transport, nous utilisons le protocole TCP.
Le protocole UDP ne nous permettais pas d'être sure que les paquets ne se soient pas perdus, se qui aurait impliqué de mettre en place un système pour éviter la désynchronisation des joueurs.
TCP nous épargnes ses problèmes en gérant le renvoie des paquets perdus.
\\

Pour la partie présentation du protocole, il faut respecter ce format:\\
\begin{lstlisting}
BEGIN <size> <data>
\end{lstlisting}
Size étant la taille de message, que nous avons arbitrairement limité à 5 octets, data le message à envoyer. Le tout doit être encodé en utf-8.
Ceci nous permet d'avoir une réception non bloquante des données. En effet, nous attendons de recevoir BEGIN, puis la taille, puis ensuite nous recevons la quantité exacte de donnée.
Avant ce mécanisme, nous avions un système d'acquittement pour éviter les réceptions bloquantes. Le serveur attendait un "OK" du client pour envoyer la suite.
Mais le client ne pouvait être sure que le serveur avait reçu le OK. Le système avec la taille des données semblait alors plus pratique à mettre en oeuvre.

\newpage
Pour communiquer avec le serveur, le client dispose de différentes commandes à envoyer au serveur.\\
\begin{description}
\item[JOIN <nickname>] Permet de rejoindre le serveur avec un pseudonyme.
\item[JOSP <healt> <kind> <nickname>] Permet de rejoindre un serveur en précisant le niveau de vie, le type de personnage et le pseudonyme. Sert pour la téléportation.
\item[MOVE <up/down/left/right>] Permet de bouger son personnage sur le serveur.
\item[DROP] Permet de lancer une bombe.
\item[QUIT] Permet d'annoncer que l'on quitte le serveur.
\end{description}
Le serveur dispose de ses propres commandes afin de communiquer avec les clients :\\
\begin{description}
\item[WELC <nick> <healt> <kind> <x> <y>] Donne au joueur ses informations tels que son type, sa vie, et (x, y) ses coordonnées de départ dans le jeu.
\item[NEWP <nickname> <healt> <kind> <x> <y>] Indique au client la création d'un joueur avec healt points de vie, de type kind, à la position (x, y)
\item[NEWF <kind> <x> <y>] Indique la création d'un fruit de type kind à la position (x, y)
\item[MOVP <nickname> <up/down/left/right] Indique que le joueur <nickname> à bougé dans une certaine direction
\item[QUIT <nickname>] Indique que le joueur <nickname> a quitté la partie
\item[DROP <nickname>] Indique que le joueur <nickname> a posé une bombe.
\item[SERVDROP <x> <y>] Indique que le serveur à posé une bombe à la position (x, y)
\item[TPSP <host> <port>] Indique au client qu'il doit se connecter au serveur <host> sur le port <port>
\end{description}
\newpage
\subsection{Implémentation du mode multijoueur}

La majeure partie du projet se déroule dans cette implémentation ; Le client et le serveur disposent chacun d'un controleur différent. Pour dialoguer, ils utilisent le protocole décrit dans la partie précédente. Les commandes reçues et envoyées sont généralement composées d'un mot-clé et de paramètres, ce qui va permettre au receveur de la décomposer puis d'appeller la méthode appropriée.
\\

Lorsqu'un client se connecte à un serveur, il lui envoie uniquement son nom d'utilisateur. Le serveur va générer un personnage, l'ajouter à son modèle, puis répondre par un message de bienvenue qui comprend les caractéristiques du personnage généré. Le serveur va également envoyer une commande à tous les autres clients pour que ceux-ci ajoutent à leur tour le personnage. En cas de nom d'utilisateur déjà utilisé, le client est simplement refusé et affiche le message d'erreur adéquat.
\\

Lorsqu'un utilisateur appuie sur une touche de déplacement, elle est directement impactée sur son jeu pour plus de fluidité. La commande est ensuite envoyée au serveur, qui va l'interprêter puis prévenir à leur tour les autres clients. Cette méthode ne provoque pas de désynchronisation, car les pertes de paquets sont gérées par TCP. Cela marche également comme ça pour toutes les autres intéractions.
\\

Le jeu se déroule donc avec chaque partie gérant le modèle de leur côté, suivant les instructions du serveur et leurs propres mouvements. Ce système n'est pas du tout sécurisé car même si le code était compilé, un client peut envoyer toutes les instructions qu'il veut. Il existe notament une faille qui sera expliquée dans la section bonus.

\subsection{Bugs et résolution de bugs}
Durant le développement de ce projet, nous avons rencontrés quelques difficultés. Un des premiers dysfonctionnement était que nous nous retrouvions bloqués à attendre des données qui avaient déjà été envoyées par le serveur. La solution à ce problème à d'abord été la mise en place d'un système d'acquittement, qui fut remplacé par un autre système permettant de connaitre la taille des données à recevoir.
\\
Nous avions aussi une erreur du côté du serveur, le serveur restait bloqué dans le select et ne mettais à jour son, model que lorsqu'il recevait les actions d'un joueur. Les explosions de bombes n'étaient bien souvent pas prises en comptes par le serveur. La solution fut de lire la documentation afin de trouver un parametre sous permettant de ne rester qu'un certain lapse de temps dans le select.\\

Nous avions également un problème avec la vie des joueurs. Quand on transmettais à un nouvel arrivant des personnages déjà présent à ajouter à son model, nous ne donnions pas la vie de ces joueurs. Le nouveau client leur donnait automatiquement 50 points de vies.\\

\subsection{Ajout de bonus}
Pour que le jeu soit plus intéressant / jouable, nous avons rajouté l'affichage de la vie en haut à gauche de l'écran. L'image de coeur est un simple pixel art et le texte est écrit avec un module de pygame. Nous avons également rajouté l'apparition de bombes aléatoirement sur le terrain, pour ajouter un peu de difficulté. 
\\

Nous avons également rajouté un teleporteur comme proposé. Le téléporteur est une nouvelle case représentée par le chiffre 3 dans le fichier de terrain, et par une version violette de la case "départ" disponible dans les fichiers image. Lorsqu'un joueur se déplace dessus, il le serveur lui transmet une nouvelle adresse et un nouveau port, où il va ensuite s'auto transférer. La faille du point de vue gameplay est qu'un client malicieux peut envoyer ces requêtes quand il le veut. Ce qui rend cette faille importante est que le joueur peut décider de ses caractéristiques (apparence et points de vie), mais cela a rendu l'implémentation plus simple.
\\

Dans cette implémentation, on conserve facilement les attributs du joueur et si le code n'est pas modifié, cela marche très bien. Si nous voulons tout de même corriger cela, nous l'aurions fait de manière à ce que ce soit le serveur qui communique avec l'autre serveur à propos des caractéristiques du joueur. Il aurait fallu, cependant, faire en sorte que personne ne se fasse passer pour le serveur et pour cela on peut penser à un système de clés publiques clés privées, puis un dictionnaire de serveurs de confiance ... Mais pour ce projet il ne nous semblait pas utile d'aller aussi loin.
\newpage

\section{Bilan}
Réaliser ce projet fut une expérience enrichissante qui nous a permis de réfléchir à des problématiques à côté desquelles nous passons souvent à côté lorsque nous développons des applications en standalone. Il fut également l'occasion de mettre en pratique par nous même des principes vus lors du semestre de Réseau.
\\

Le résultat que nous avons fourni nous satisfait et nous sentons que si une demande de nouvelle fonctionnalités surgissait, ce ne serait pas à contre coeur que nous nous remettrions à travailler dessus, même si il fallait pour cela revenir sur la majorité du code. En attendant, nous avons eu le plaisir de tester ce jeu auquel nous avons rajouté la dimension multijoueur depuis un environnement local, à distance sur un ordinateur personnel puis sur un vrai serveur, afin de constater qu'il est pleinement opérationnel et jouable.
\end{document}