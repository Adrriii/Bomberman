\documentclass[12pt]{article}
\usepackage[T1]{fontenc}
\usepackage[utf8]{inputenc}
\usepackage[francais]{babel}
\usepackage{listings}

\title{Rapport groupe TM7B}

\author{NAVARRO Robin, BOITELLE Adrien}


\begin{document}
\maketitle

\newpage
\tableofcontents
\newpage

\section{Introduction}
Pour réaliser ce projet, nous nous sommes beaucoup appuyé à la fois sur le jeu existant, sur son architecture et sur le code du serveur IRC précédemment réalisé en TP. Dans ce rapport nous allons présenter le travail effectué, les méthodes utilisées, le résultat produit ainsi que les possibles améliorations.

\section{Vue d'ensemble fonctionnelle}

Cette section est destinée aux potentiels utilisateurs afin de présenter le projet, guider l'utilisation et alerter des problèmes connus.

\subsection{Utilisation}

Des scripts sont à disposition pour tester les mécanismes de ce projet:
\begin{lstlisting}[language=bash]
  $ ./run_servers.sh
  $ ./play_1.sh
  $ ./play_2.sh
\end{lstlisting}

Pour démarrer un serveur manuellement ou en rejoindre un, les commandes de base fonctionnent : 
\begin{lstlisting}[language=bash]
  $ python bomber_server.py {port} {map_path}
  $ python bomber_client.py {host} {port} {username}
\end{lstlisting}

Attention cependant, pour que le téléporteur fonctionne, il est nécessaire d'utiliser les ports 7777 et 7778.
\subsection{Fonctionnalités}
Une fois connecté, un cadre rouge permet d'identifier le personnage contrôlé. Une immunité temporaire est accordée au nouveaux venus. Chaque explosion retire 10 point de vie. Chaque fruit mangé en rend 10. Les points de vie du joueur sont visibles en haut à gauche de l'écran. Le pavé directionnel permet de déplacer le personnage dans le jeu. Appuyer sur espace permet de poser une bombe à l'emplacement du joueur. Il y a un délai pour pouvoir poser une autre bombe.
Une case particulière, représentée en rose dans le jeu, permet de se téléporter vers une autre serveur.
Dans le cas où cet autre serveur ne serai pas accessible, la téléportation ne se fait pas.
Des bombes apparaissent aléatoirement sur le jeu, afin d'augmenter la difficulté.

\subsection{Problèmes connus}
Malgré nos efforts, quelque problèmes persistent, et quelques fonctionnalités manquent.\\
Au lancement du jeu, aucune vérification n'est faite afin de vérifier que la grille de jeu soit correcte.\\
En case de déconnexion du serveur, les client sont fermés. Il n'y a pas de mécanisme permettant de relancer le jeu.
Par contre, si les joueurs ont utilisés un téléporteur et se trouvent du bon côté, il peuvent continuer à jouer. Le téléporteur ne fonctionnera tout simplement plus.
\\
Au cours de développement, il nous est arrivé de rencontré quelques fonctionnements anormaux, mais impossible à reproduire.\\
Par exemple, il est arrivé une fois que les points de vies d'un joueur soient différents de 10 unités entre les clients et le serveur. Nous n'avons pas réussis à reproduire cet effet.\\
Ainsi nous ne pouvons que spéculer sur la nature du dysfonctionnement.
Il se pourrait qu'un joueur arrive à esquiver une bombe juste avant son explosion.\\
Mais, avec une latence élevée, il se peut qu'au moment de l'explosion, les autres clients n'aient pas reçu le mouvement du joueur.\\
Il en découlerai alors un décalage de point de vie.
La solution envisagé est alors d'avoir tous les clients esclaves du serveur, qui imposerait le statut du jeu et des joueurs. Mais cette solution impliquait une refonte importante du modèle, afin qu'il soit différent pour les joueurs et les serveurs.
\newpage

\section{Architecture et Implémentation}
\subsection{Étude du modèle existant}

\subsection{Protocole réalisé}

\subsection{Implémentation du mode multijoueur}

\subsection{Bugs et résolution de bugs}

\subsection{Ajout de bonus}
\newpage

\section{Bilan}

\end{document}