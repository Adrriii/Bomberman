\documentclass[12pt]{article}
\usepackage[T1]{fontenc}
\usepackage[utf8]{inputenc}
\usepackage[francais]{babel}
\usepackage{listings}

\title{Rapport groupe TM7B}

\author{NAVARRO Robin, BOITELLE Adrien}


\begin{document}
\maketitle

\newpage
\tableofcontents
\newpage

\section{Introduction}
Pour réaliser ce projet, nous nous sommes beaucoup appuyé à la fois sur le jeu existant, sur son architecture et sur le code du serveur IRC précédemment réalisé en TP. Dans ce rapport nous allons présenter le travail effectué, les méthodes utilisées, le résultat produit ainsi que les possibles améliorations.

\section{Vue d'ensemble fonctionnelle}

Cette section est destinée aux potentiels utilisateurs afin de présenter le projet, guider l'utilisation et alerter des problèmes connus.

\subsection{Utilisation}

Des scripts sont à disposition pour tester les mécanismes de ce projet:
\begin{lstlisting}[language=bash]
  $ ./run_servers.sh
  $ ./run_players.sh
\end{lstlisting}
Si ceux-ci ne fonctionnent pas, remplacer python3 par python. Sinon, vérifier que pygame est bien installé.

Pour démarrer un serveur manuellement ou en rejoindre un, les commandes de base fonctionnent : 
\begin{lstlisting}[language=bash]
  $ python bomber_server.py {port} {map_path}
  $ python bomber_client.py {host} {port} {username}
\end{lstlisting}

Attention cependant, pour que le téléporteur fonctionne, il est nécessaire d'utiliser les ports 7777 et 7778.
\subsection{Fonctionnalités}
Une fois connecté, un cadre rouge permet d'identifier le personnage contrôlé. Une immunité de 5 secondes est accordée au nouveaux venus. Chaque explosion retire 10 point de vie et chaque fruit mangé en rend 10. Les points de vie du joueur sont visibles en haut à gauche de l'écran et le pavé directionnel permet de déplacer le personnage dans le jeu. Appuyer sur espace permet de poser une bombe à l'emplacement du joueur (il y a un délai de 2 secondes avant de pouvoir poser une autre bombe).
\newpage
Une case particulière, représentée en violet dans le jeu, permet de se téléporter vers une autre serveur.
Dans le cas où cet autre serveur ne serait pas accessible, la téléportation ne se fait pas.
Des bombes apparaissent aléatoirement sur le terrain, afin d'augmenter la difficulté.

\subsection{Problèmes connus}
Malgré nos efforts, quelque problèmes persistent.\\
Au lancement du jeu, aucune vérification n'est faite afin de vérifier que la grille de jeu corresponde à celle du serveur. Nous avons pensé à effectuer une verification de l'intégrité du fichier du terrain puis dans le cas d'un echec à son remplacement, mais cela semblait un peu en dehors des attentes.
\\

En cas de déconnexion du serveur, les clients sont immédiatement fermés. Il n'y a pas de mécanisme permettant de relancer le jeu automatiquement, ou de reprendre la partie.
Par contre, si les joueurs ont utilisé un téléporteur et se trouvent sur un serveur qui n'a pas fermé, il peuvent continuer à jouer. Le téléporteur ne fonctionnera tout simplement plus tant que l'autre serveur ne sera pas à nouveau disponible.
\\

Au cours de développement, il nous est arrivé de rencontrer quelques fonctionnements anormaux, mais impossible de les reproduire.\\
Par exemple, il est arrivé une fois que les points de vies d'un joueur soient différents de 10 unités entre les clients et le serveur. Nous n'avons pas réussi à reproduire cet effet, nous ne pouvons donc que spéculer sur la nature du dysfonctionnement. \\

L'hypothèse la plus probable est qu'un joueur esquive une bombe in-extremis de son point de vue, mais que son mouvement ne soit pas envoyé au serveur et aux autres clients à temps à cause de la latence.
La solution envisagée serait que les clients ne décident plus de retirer de la vie aux autres joueurs, mais que ce soit le serveur qui leur en donnent l'indication.
\newpage

\section{Architecture et Implémentation}
Nous avons implémenté nos modifications en tentant au maximum de respecter la structure existante. L'architecture est donc restée identique à celle proposée.

\subsection{Étude du modèle existant}
Le jeu tel que nous l'avons récupéré utilise le modèle Modèle Vue Controleur. Le controleur avait été modifié pour réaliser une intéraction entre clients et serveurs, tandis que le modèle est resté identique. La vue n'a également pas été modifiée, mais elle n'est pas invoquée complètement par le serveur.
\\

Pour son interface graphique, ce jeu utilise pygame qui est une librairie libre et open source destinée à créer des jeux en python, reposant sur la librairie SDL.
\\

Pour générer les terrains, un format texte est utilisé. Il est décodé dans la vue pour en faire un objet map. Pour que le jeu soit opérant, il faut que chacun ait les mêmes fichiers. Aucune vérification n'est imposée donc il est tout à fait possible qu'un joueur rejoigne une partie avec un fichier de terrain différent, à ses risques et périls.
\subsection{Protocole réalisé}

\subsection{Implémentation du mode multijoueur}

\subsection{Bugs et résolution de bugs}

\subsection{Ajout de bonus}
\newpage

\section{Bilan}

\end{document}